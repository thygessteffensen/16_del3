\subsection{UseCases}
%1. UseCase
\begin{center}
    \begin{tabular}{ | m{10em} | m{10cm}| }
        \hline
            UseCase Section: Start af spil & Comment\\
        \hline
            Scope & Monopoly spil af IOOuterActive\\
        \hline
            Level & User-goal\\
        \hline
            Primær Aktør & IOOuterActive\\
        \hline
            Stakeholder og interessenter
            & IOOuterActive er interesseret i at spillerne skal kunne starte spillet\\
        \hline
            Forudsætninger & Spillet er installeret på enheden\\
        \hline
            Success guaranti & Spillet starter og spillerne bliver sendt videre til opsætning\\
        \hline
    \end{tabular}
\end{center}

%2. UseCase
\begin{center}
\begin{tabular}{ | m{10em} | m{10cm}| }
        \hline
            UseCase Section: Opsætning af spil & Comment\\
        \hline
            Scope & Monopoly spil af IOOuterActive\\
        \hline
            Level & User-goal\\
        \hline
            Primær Aktør & Spillerne\\
        \hline
            Stakeholder og interessenter & Spillerne er interesserede i at kunne vælge antal spillere og deres brikker\\
        \hline
            Forudsætninger & Spillet er startet op, og spillerne har nu mulighed for at vælge antal spillere og ønskede brikker\\
        \hline
            Success guaranti & Der er blevet valgt antallet af spillere, og hver spiller har valgt sin brik, herefter er spillet klar til at blive spillet\\
        \hline
    \end{tabular}
\end{center}
%3. UseCase
\pagebreak
Fully dressed UseCase:
\begin{center}
\begin{tabular}{ | m{10em} | m{10cm}| }
        \hline
            UseCase Section: Spillerne slår med terningerne & Comment\\
        \hline
            Scope & Monopoly spil af IOOuterActive\\
        \hline
            Level & User-goal\\
        \hline
            Primær Aktør & Spillerne\\
        \hline
            Stakeholder og interessenter & Spillerne er interesseret i at kunne trykke på en knap, og få et billede af to terninger med tilfældige værdier\\
        \hline
            Forudsætninger & Spillet er startet op, og spillerne har valgt antallet spillere og deres ønskede brikker\\
        \hline
            Success guaranti & Der er blevet valgt antallet af spillere, og hver spiller har valgt sit navn, herefter er spillet klar til at blive spillet\\
        \hline
            Hoved succes scenarie & Spillerne får udgivet en værdi af to terninger, og lander derefter på et felt\\
        \hline
            Alternative udfald & Negative udfald:\\
                & -	IOOuterActive har opdateret spillet, og derved opstår der en fejl når spillerne slå med terningerne, der kan ende i at der ikke bliver slået to terninger\\
                & -	Systemet blokerer for en spillers tur\\
                & -	En spiller hopper fra/på, og derved skal spillet startes om\\
        \hline
            Specielle krav
            & -	Enheden som spillet kører på skal være kompatibel med Java\\
            & -	Spillerne skal kunne interagere med GUI’en ved brug af mus eller touch\\
            & -	Der skal være plads på enheden til at kunne hente spillet\\
        \hline
            Hyppighed & Hver tur bliver der slået med terninger\\
        \hline
    \end{tabular}
\end{center}
%4. UseCase
\begin{center}
\begin{tabular}{ | m{10em} | m{10cm}| }
        \hline
            UseCase Section: Spiller køber et felt & Comment\\
        \hline
            Scope & Monopoly spil af IOOuterActive\\
        \hline
            Level & User-goal\\
        \hline
            Primær Aktør & Spillerne\\
        \hline
            Stakeholder og interessenter & Spillerne er interesseret i at købe det aktuelle felt\\
        \hline
            Forudsætninger & Spillet er i gang og en spiller har slået med terningerne\\
        \hline
            Success guaranti & Spilleren køber og ejer nu feltet\\
        \hline
    \end{tabular}
\end{center}

%5. UseCase
\begin{center}
\begin{tabular}{ | m{10em} | m{10cm}| }
        \hline
            UseCase Section: Spiller lander på et ejet felt & Comment\\
        \hline
            Scope & Monopoly spil af IOOuterActive\\
        \hline
            Level & User-goal\\
        \hline
            Primær Aktør & Spillerne\\
        \hline
            Stakeholder og interessenter & ****\\
        \hline
            Forudsætninger & Spillet er i gang og en spiller lander på et felt som er ejet af en anden spiller\\
        \hline
            Success guaranti & En spiller lander på et felt der er ejet af en anden spiller og får derfor en negativ effekt\\
        \hline
    \end{tabular}
\end{center}

%6. UseCase
\begin{center}
\begin{tabular}{ | m{10em} | m{10cm}| }
        \hline
            UseCase Section: En spiller taber & Comment\\
        \hline
            Scope & Monopoly spil af IOOuterActive\\
        \hline
            Level & User-goal\\
        \hline
            Primær Aktør & Terningerne\\
        \hline
            Stakeholder og interessenter & Spillerne er interesseret i at deres balance ikke når 0, og dermed taber\\
        \hline
            Forudsætninger & Spillerne har slået med terningerne\\
        \hline
            Success guaranti & En spiller lander på 0, og er ude af spillet\\
        \hline
    \end{tabular}
\end{center}

%7. UseCase
\begin{center}
\begin{tabular}{ | m{10em} | m{10cm}| }
        \hline
            UseCase Section: Spillet afsluttes & Comment\\
        \hline
            Scope & Monopoly spil af IOOuterActive\\
        \hline
            Level & User-goal\\
        \hline
            Primær Aktør & IOOuterActive\\
        \hline
            Stakeholder og interessenter & IOOuterActive er interesserede i at programmet viser en vinder og afsluttes\\
        \hline
            Forudsætninger & Alle spillere undtagen en, har fået en balance på 0\\
        \hline
            Success guaranti &Spillet viser en vinder og kan derefter afsluttes\\
        \hline
    \end{tabular}
\end{center}