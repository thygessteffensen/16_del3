\section{Dokumentation}
\subsection{Forklar hvad arv er}

\subsection{Forklar hvad abtract betyder}

\subsection{Fortæl hvad det hedder hvis alle fieldklasserne har en landOnField metode der gør noget forskelligt}
Hvis alle fieldklasserne gør brug af den samme landOnField metode er det fordi denne metode er en super metode nedarvet til de forskellige subklasser,
dette gør at alle klasserne kan gøre brug af samme metode. Eksempeltvis, hvis vi har en masse dyr som klasser, kat, hund, kanin etc. så kan de alle nedarve super metoden eat() fra 
superklassen som hedder SurvivalRequirements, eftersom disse er metoder alle dyrene får brug for, så vil det give mening at lave det til en superklasse med supermetoder
, så der holdes lav kobling og høj kohæsion, samt undgås kopiring af kode og høj mulighed for genbrug.
\subsection{Dokumentation for test med screenshots}
    \subsection{JUnit test}
        JUnit test er en autonomisreret testmetode. Her skriver/koder man selv en test, som tester java kode. Oftest opbygger man JUnit test ud fra Java klasser.
        Der er mange måder hvorpå man kan bruge JUnit testen. Man kan både skrive testen inden, man kan skrive den efter, man kan lave den på baggrund af indsigt i koden eller uden nogen form til kendskab af programkoden. Det to sidst nævnte kaldes Black- og Whitebox test.
        Her er et eksempel på et stykke udført JUnit test fra vores spil:
        
    \subsection{Positiv negativ test}

    \subsection{Black- og Whitebox test}

\subsection{Dokumentation for overholdt GRASP}
