\section{Design}
    Herunder ses en række 'design steps', som skal hjælpe os med at lave Monopoly 
    Junior spillet.

    \subsection{Klasse diagram}
            \begin{figure}[h]
                \advance\leftskip-3cm
                \includegraphics[width=20cm]{fig/Designklassediagram(3).jpg}
                \caption{Klasse diagram tegnet i UMLet}
            \end{figure}
        Klasse diagrammet bygger på vores umiddelbare overvejelser, såvel som vores use case's.
        Dette er for at illustere sammenspillet mellem vores klasser og deres associationer.
        Dog skal det siges, at dette er en skitse og den aktuelle programmering kan variere heraf.
        \pagebreak
    
    \subsection{Sekvensdiagram}
        \begin{figure}[h]
            \advance\leftskip5cm
            \includegraphics[width=12cm]{fig/Sekvensdiagram(1).jpg}
            \caption{Sekvensdiagram tegnet i MagicDraw}
        \end{figure} 
    Vi har her lavet et sekvensdiagram, der skal skabe et overblik over hvordan aktøren, her spilleren,
    kommunikerer med spillet.
\pagebreak
    
    \subsection{System sekvensdiagram}   
        \begin{figure}[h]
            %\advance\leftskip+2cm
            \includegraphics[width=10cm]{fig/SSD.jpg}
            \caption{Systemsekvensdiagram tegnet i MagicDraw}
        \end{figure}
    Vi har her lavet et systemsekvensdiagram, for at forhøje gennemsigtigheden ved bruge af
    'chanceCard' klassen.
\pagebreak
    
    \subsection{Domænemodel}
        \begin{figure}[h]
            \advance\leftskip-3cm
            \includegraphics[width=20cm]{fig/domainemodel.PNG}
            \caption{Domænemodel tegnet i UMLet}
        \end{figure}
    Ved hjælp af domænemodellen vil vi trække paraleller mellem den virkelige
    verden og programmeringen. Domænemodellen er en visuel repræsentation af 
    konceptklasser og 'objekter fra den virkelige verden'. Ved hjælp af denne
    kan vi også oplyse kunden om, hvad vi vil lave.
